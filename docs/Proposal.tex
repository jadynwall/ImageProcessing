\documentclass[conference]{IEEEtran}
\usepackage{cite}
\usepackage{amsmath,amssymb,amsfonts}
\usepackage{algorithmic}
\usepackage{graphicx}
\usepackage{textcomp}
\usepackage{xcolor}
\def\BibTeX{{\rm B\kern-.05em{\sc i\kern-.025em b}\kern-.08em
    T\kern-.1667em\lower.7ex\hbox{E}\kern-.125emX}}
\begin{document}

\title{Project Proposal - Blackjack Strategist\\
{\footnotesize COMPENG 4TN4: Image Processing}
}

\author{\IEEEauthorblockN{Ben Sun}
\IEEEauthorblockA{\textit{Dept. of Computing and Software} \\
\textit{McMaster University}\\
Hamilton, Canada \\
sunb26@mcmaster.ca}
\and
\IEEEauthorblockN{Jadyn Wall}
\IEEEauthorblockA{\textit{Dept. of Computing and Software} \\
\textit{McMaster University}\\
Hamilton, Canada \\
wallj4@mcmaster.ca}
\and
\IEEEauthorblockN{Matthew Wilker}
\IEEEauthorblockA{\textit{Dept. of Computing and Software} \\
\textit{McMaster University}\\
Hamilton, Canada \\
wilkem1@mcmaster.ca}
}

\maketitle

\begin{abstract}
    This project aims to develop a computer vision-based system to assist blackjack
     players in making mathematically optimal gameplay decisions. By processing images
     of the blackjack table to identify the dealer and player cards, the system
      provides the move most statistically likely to win using a predefined lookup
      table.
\end{abstract}

\begin{IEEEkeywords}
image filtering, image segmentation, object recognition
\end{IEEEkeywords}

\section{Introduction}
The overall goal of this project is to aid players in making the mathematically correct
decision when playing the card game blackjack \cite{baldwin_optimum_1956}. This will require inputting an
image of the table and cards from the player’s perspective at any time a decision needs to
be made by the player. Various image processing methods will be applied to the input image
to determine what cards both the player and dealer have (see ‘Methods Used’ section). Once
the cards belonging to each player have been established, the optimal move for the player
will be displayed. This will be determined via a lookup table dictating the mathematically
optimal move (hit, stand, double, split, surrender, or take insurance) for any given
circumstance \cite{cai_house_2022}.

\section{Expected Results}

An application that takes in an image of any situation in blackjack (from the player’s
perspective), and outputs their mathematically optimal move. The application should return
the correct move 99.9\% of the time, given the cards are recognizable by the optical character
recognition (OCR) algorithm. 

\section{Methods Used}
OCR will need to be used to identify the alphanumeric value of each card within the image.
Before OCR can be used, the image needs to be preprocessed to remove any noise that may hinder
the performance of OCR.

\subsection{Image Preprocessing}
To begin, the image will need to be converted to grayscale for easier computation given colour
is not required for this application. Next, the noise from the image needs to be removed. A
Gaussian filter will be used to reduce blur caused by Gaussian noise which is anticipated to be
the most prominent type of noise \cite{boncelet_image_2009}. Other forms of noise may appear during implementation and
will be addressed accordingly on a case-by-case basis. The images will be acquired via an iPhone,
and may therefore be susceptible to shadows. Shadows will need to be removed to help with the
segmentation step afterwards \cite{finlayson_removing_2002}. Next, the card value in the corner must be isolated so the image
will be segmented to remove any unnecessary information. Once the image has been segmented, it
will need to be skew corrected, which is necessary because if the image of the value is skewed,
it could be misinterpreted by the OCR algorithm (e.g., 6 vs 9)  \cite{sarfraz_novel_2005}.

\subsection{OCR}
The preprocessed image will be inputted into an OCR model to identify the card values and
convert them to text.

\subsection{Decision Making}
The dealer and player card values are compared to a lookup table to determine what move the player
should make.

\bibliographystyle{IEEEtran}
\bibliography{references}

\end{document}
